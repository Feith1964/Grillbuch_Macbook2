%!TEX root = Vorlage_Buch.tex
\chapter{Gemüse ein unterschätzter Darsteller in der Küche}\label{Chapter3}

\lettrine[lines=3]{G}{emüse mach Spass, wetten dass?} In diesem Kapitel möchte ich euch zeigen, dass Gemüse zu Schade ist um als Beilage oder zu Kaninchenfutter degradiert wird sondern richtiges Soulfood, nicht nur für Veggies, ist. Es ist bunt, es ist vielseitig, es ist knackig (wenn es nicht getötet wird) und es ist vor allem vielseitig. Lassen wir uns auf das Experiment Gemüse ein.

\section{Gemüse als Beilage}

\subsection{Weißer Spargel, gedämpft}

\paragraph{Geräte}

\begin{itemize}[noitemsep]
	\item Holzkohlegrill
	\item Gasgrill
	\item Kamado Grill
\end{itemize}

\paragraph{Zutaten}

\begin{itemize}[noitemsep]
	\item 1,5 kg Spargel, geschält
	\item Saft einer Zitrone
	\item Olivenöl, genau soviel wie Zitronensaft
	\item 1 gute Prise Salz
	\item 1 gute Prise Zucker
	\item 3 Bögen Backpapier
	\item Aluminiumfolie
\end{itemize}
	
\paragraph{Zubereitung}
Den Zitronensaft, das Öl, das Salz und den Zucker mit eine kleinen Schneebesen verrühren. Je ein Drittel der Spargel auf einen Bogen Backpapier nebeneinander legen und mit einem drittel der Marinade beträufeln. Das Backpapier einschlagen, sodass dichte Päckchen entstehen. Diese Päckchen in Alufolie einschlagen und bei ca. 200°C 20-25 Minuten auf der indirekten Zone bissfest garen. Die Spargel mit Lavendelbutter Kapitel~\ref{Chapter6} \vref{LavButter} bestreichen und servieren. Die Aromatik ist unbeschreiblich.

