%!TEX root = Vorlage_Buch.tex
\chapter{Beilagen und Snacks}\label{Chapter4}
\lettrine[lines=3]{BBQ}{bedeutet für mich} nicht nur Fleisch oder 'ne Rote im Brötchen. BBQ ist kochen und im Idealfall auch genießen im 
Freien. Fast alles was in der Küche im Haus zubereitet werden kann, kann auch in der Außenküche zubereitet werden.  Deshalb werde ich 
hier auch einige Rezepte vorstellen die von Oma gekocht wurden, bevor es jemanden gab der Barbecue (/ˈbɑː.bɪ.kjuː/)  aussprechen konnte.

Unter dem Punkt "`Ä Gosch voll"' werden wir Rezepte für kleine Happen kennenlernen.  Diese Häppchen sind mundgerecht und können als Amuse Gueule, 
zwischen den Gängen, als Dessert oder einfach als Fingerfood gereicht werden. In diesen Happen wird alles verpackt, das benötigt um Kleines ganz groß werde zu 
lassen. Also gehen wir es an.  

\section{Ä Gosch voll,}
besser beschreiben kann man die Kleinigkeiten nicht. Es ist wie im wirklichen Leben, die kleinen Dinge machen die meiste Arbeit, sorgen aber auch für den Größten 
Wow-Effekt. Daher bitte ein wenig Zeit einplanen, oder so viel möglich am Vortag vorbereiten. 

\subsection{Tiroler Schlutzkrapfen}

Schlutzkrapfen sind eine Nudelspezialität aus Tirol und werden aus eine Mischung von Roggen- und Weizenmehl hergestellt. Der Name leitet sich von "`Schluzen"' 
ab, was soviel wie Rutschen oder Gleiten bedeutet.

\paragraph{Geräte}

\begin{itemize}[noitemsep]
	\item Teigmaschine oder Hände
	\item Nudelmaschine
	\item Holzkohlegrill, Gasgrill oder Keramikgrill
\end{itemize}

\paragraph{Zutaten für die Schlutzkrapfen}

\begin{itemize}[noitemsep]
	\item 125 g Weizenmehl (hier Type 405)
	\item 125 g Roggenmehl (hier Type 997)
	\item 2 Eier
	\item 30 g Butter
	\item 2 L Wasser
	\item 1 TL Salz (hier Himalaya Salz)
\end{itemize}

\paragraph{Zubereitung}

Die beiden Mehlsorten mit dem Salz in einer Schüssel mischen. Die Butter zerlassen und mit den Eiern, dem Wasser hinzugeben. Das 
ganze mit der Küchenmaschine oder den Händen zu einem glatten Teig kneten. In eine Küchentuch einschlagen und 30 Minuten ruhen 
lassen.

Den Nudelteig auf 2-3mm Stärke ausrollen und mit einem runden Ausstecher oder einem Glas ca. 8 cm durchmessenden Kreise 
ausstechen. Diese mit der Kartoffelfüllung füllen, auf eine Seite benetzen, zusammenklappen und die Ränder mit den Fingern unter Druck 
verschließen.

Die "`Schlutzer"' in Salzwasser 3-4 Minuten leicht köcheln lassen. Die Teigtaschen auf gewässerte Holzspiese spießen und und bei 
indirekter Hitze bei ca. 180 °C grillen, bis leicht gebräunt und eine wenig knusprig sind. Auf einem kleinen Teller oder Gourmetlöffel 
anrichten und mit der Orangen-Chili-Butter beträufeln. Ein unglaubliches Geschmackserlebnis erwartet euch.

\paragraph{Zutaten für die Füllung}

\begin{itemize}[noitemsep]
	\item 400g vorwiegend festkochende Kartoffeln
	\item 1 Bund Minze
	\item 250 g Frischkäse
	\item 2 Knoblauchzehen
	\item Salz
	\item Cayennepfeffer
\end{itemize}

\paragraph{Zubereitung}

Kartoffeln ca. 25 Minuten kochen, abkühlen lassen und pellen. Die Kartoffeln noch warm durch eine Kartoffel- oder Spätzlepresse 
pressen und auskühlen lassen. Die Minze waschen, trocken schütteln und fein schneiden. Die Knoblauchzehen Schälen und sehr fein 
hacken, nicht pressen da sie bitter werden. Die Minze, den Knoblauch und 250 g Frischkäse zu den Kartoffeln geben, alles durchmischen 
und mit Salz und Cayennepfeffer abschmecken.

\paragraph{Finale Zubereitung der Schlutzkrapfen}

Den Nudelteig auf 2-3mm Stärke ausrollen und mit einem runden Ausstecher oder einem Glas ca. 8 cm durchmessende Kreise 
ausstechen. Diese mit der Kartoffelfüllung füllen, auf einer Seite benetzen, zusammenklappen und die Ränder mit den Fingern unter 
Druck verschließen.

Die "`Schlutzer"' in Salzwasser 3-4 Minuten leicht köcheln lassen. Die Teigtaschen auf gewässerte Holzspiese spießen und bei 
indirekter Hitze bei ca. 180 °C grillen, bis sie leicht gebräunt und ein wenig knusprig sind. Auf einem kleinen Teller oder Gourmetlöffel 
anrichten und mit Orangen-Chili-Butter (Rezept siehe 
Kapitel~\ref{Chapter6} \vref{OrangenChili}) beträufeln. Ein unglaubliches Geschmackserlebnis erwartet euch.

\subsection{Albóndigas}

Die Herkunft der Albóndigas en salsa, der spanischen Hackbällchen-Tapa lässt sich den arabischen Mauren zuschreiben. Sie brachten 
die al-búnduqa (arabisch für die Kugel) im 13. Jahrhundert nach Spanien. Im Morgenland wurden die Bällchen traditionell noch aus Lamm 
gerollt. In anderen Ländern kamen mit der Zeit Varianten aus Rind, Schwein, Wild, Geflügel oder Fisch dazu. Ich verwende gemischtes 
Hackfleisch aus Rind \& Schwein, um die Vorzüge der zwei Fleischsorten zu vereinigen. Das Rinderhack sorgt für eine kernigen Biss 
währen das fettere Schweinehack die Hackfleischbällchen saftig werden lässt.

\paragraph{Geräte}

\begin{itemize}[noitemsep]
	\item Holzkohlegrill, Gasgrill oder Keramikgrill
\end{itemize}

\paragraph{Zutaten}

\begin{itemize}[noitemsep]
	\item 400 g gemischtes Hackfleisch
	\item 1 kleine Zwiebel
	\item 1 Knoblauchzehe
	\item 1 TL Pimenton picante (Räucherpaprika)
	\item 1 TL Cumin geröstet und geschrotet
	\item 1 EL glatte Petersilie
	\item 1 Ei, Größe M
	\item 1 kleine Tasse Panko (60 g), alt. trockenes Brot oder Semmelbrösel
	\item Salz, Pfeffer, Muskatnuss
	\item  Anschovis, gesalzen (nach Geschmack), die Anchovis können auch weggelassen werden
	\item 1 Prise Zimt
	\item 1/2 Kaschmir-Chili-Pulver
\end{itemize}

\paragraph{Zubereitung}

Den Grill mit einer direkten und indirekten Zone vorbereiten und auf eine Temperatur von 170°C bis 200°C einstellen.

Zwiebel und Knoblauch möglichst fein würfeln. Petersilie fein hacken. Cumin ohne Fett im Pfännchen anrösten und in der Gewürzmühle 
oder im Mörser grob mahlen. Die Anchovis ebenfalls hacken, werden keine Anchovis verwendet muss die Masse ein wenig stärker 
gesalzen werden. Altbackenes Brot klein schneiden und fein zerbröseln. Panko oder Semmelbrösel einfach direkt verwenden.

Saubere Hände anfeuchten und das Hackfleisch mit den geschnittenen Zutaten, Ei, Panko und den Gewürzen gründlich kneten. Zu 
kleinen Bällchen von ca. 4 cm Durchmesser rollen.
Die Bällchen auf gewässerte Holzspieße spießen und auf dem Grill bei direkter Hitze scharf angrillen und auf der indirekten Zone gar 
ziehen. Die Bällchen auf einem Gourmetlöffel oder einem kleinen Teller mit der Salsa de las Albóndigas (Rezept siehe 
Kapitel~\ref{Chapter6} \vref{SalsaAlbondigas}) anrichten und mit gehackter Petersilie bestreuen.

\subsection{Wachtelspiegelei auf marinierter Wachtelbrust}

\paragraph{Geräte}

\begin{itemize}[noitemsep]
	\item Gasgrill mit Plancha
\end{itemize}
	
\paragraph{Zutaten}

\begin{itemize}[noitemsep]
	\item 1 Wachtelbrüstle/ Person
	\item 1 Wachtelei/ Person
	\item Saft von einer Zitrone nach Bedarf
	\item Salz
	\item Pfeffer
	\item 1 Zweig Bio-Lavendel
	\item Wasser nach Bedarf
\end{itemize}

Die Wachtelbrüstchen über Nacht mit einer Marinade aus Zitronensaft, Bio-Lavendel und Wasser marinieren. Die Brüstchen trocken tupfen , mit Salz und Pfeffer würzen und  auf der Plancha bei ca. 200°C anbraten. Parallel dazu die Wachtelspiegeleier zubereiten. Alles auf einem Gourmetlöffel anrichten und mit ein wenig Lavendel garnieren.

\subsection{ Gegrillte Wachtelkeulen mit Chili-Butter }

\paragraph{Geräte}

\begin{itemize}[noitemsep]
	\item Gasgrill mit Plancha
\end{itemize}

\paragraph{Zutaten}

\begin{itemize}[noitemsep]
	\item 1 Wachtelkeulen/ Person
	\item Kräuterbutter
	\item Salz
	\item Pfeffer
	\item Geklärte Butter
	\item Pulver vom Kaschmir-Chili
\end{itemize}

Die Wachtelkeulen mit Kräuterbutter einreiben und die Kräuterbutter auch unter der Haut verteilen. Die Keulen werden auf der Plancha scharf angegrillt bis sich Röststoffe bilden, danach auf der indirekten Zone fertig gegart. Die geklärte Butter auf schwacher Hitze schmelzen und das Chili-Pulver einrühren. Das Pulver der Kaschmir-Chilis sorgt für eine tolle rote Färbung und hat einen guten  Geschmack bei moderater Schärfe. Die Keulen auf einem kleinen Teller anrichten und mit der Chili-Butter beträufeln.

\section{Beilagen}

\subsection{Kartoffel-Wedges}

\paragraph{Geräte}

\begin{itemize}[noitemsep]
	\item Kamado-Grill
\end{itemize}

\paragraph{Zutaten}

\begin{itemize}[noitemsep]
	\item 500g Kartoffeln, vorwiegend festkochend
	\item 1 TL Bratkartoffelgewürz/pro 100g Kartoffeln
	\item Olivenöl
\end{itemize}

\paragraph{Zubereitung}

Die ungeschälten Kartoffeln kaltem Wasser waschen und gegebenenfalls abbürsten. Danach die Kartoffeln in ca. 1,5 cm dicke Spalten schneiden. Spalten mit einer reichlichen Menge Olivenöl beträufeln und mischen. Mein Bratkartoffelgewürz, Rezept siehe Kapitel~\ref{Chapter6} \vref{Bratkartoffelgewürz}, darüber streuen und alles nochmals mischen, bis die Kartoffeln gleichmäßig gewürzt sind. Auf dem Kamado-Grill mit einem indirekten Setting bei ca. 200°C grillen bis sie gleichmäßig gebräunt und knusprig sind, ca. 30 -35 Minuten.